% Options for packages loaded elsewhere
% Options for packages loaded elsewhere
\PassOptionsToPackage{unicode}{hyperref}
\PassOptionsToPackage{hyphens}{url}
\PassOptionsToPackage{dvipsnames,svgnames,x11names}{xcolor}
%
\documentclass[
  letterpaper,
  DIV=11,
  numbers=noendperiod]{scrartcl}
\usepackage{xcolor}
\usepackage{amsmath,amssymb}
\setcounter{secnumdepth}{-\maxdimen} % remove section numbering
\usepackage{iftex}
\ifPDFTeX
  \usepackage[T1]{fontenc}
  \usepackage[utf8]{inputenc}
  \usepackage{textcomp} % provide euro and other symbols
\else % if luatex or xetex
  \usepackage{unicode-math} % this also loads fontspec
  \defaultfontfeatures{Scale=MatchLowercase}
  \defaultfontfeatures[\rmfamily]{Ligatures=TeX,Scale=1}
\fi
\usepackage{lmodern}
\ifPDFTeX\else
  % xetex/luatex font selection
\fi
% Use upquote if available, for straight quotes in verbatim environments
\IfFileExists{upquote.sty}{\usepackage{upquote}}{}
\IfFileExists{microtype.sty}{% use microtype if available
  \usepackage[]{microtype}
  \UseMicrotypeSet[protrusion]{basicmath} % disable protrusion for tt fonts
}{}
\makeatletter
\@ifundefined{KOMAClassName}{% if non-KOMA class
  \IfFileExists{parskip.sty}{%
    \usepackage{parskip}
  }{% else
    \setlength{\parindent}{0pt}
    \setlength{\parskip}{6pt plus 2pt minus 1pt}}
}{% if KOMA class
  \KOMAoptions{parskip=half}}
\makeatother
% Make \paragraph and \subparagraph free-standing
\makeatletter
\ifx\paragraph\undefined\else
  \let\oldparagraph\paragraph
  \renewcommand{\paragraph}{
    \@ifstar
      \xxxParagraphStar
      \xxxParagraphNoStar
  }
  \newcommand{\xxxParagraphStar}[1]{\oldparagraph*{#1}\mbox{}}
  \newcommand{\xxxParagraphNoStar}[1]{\oldparagraph{#1}\mbox{}}
\fi
\ifx\subparagraph\undefined\else
  \let\oldsubparagraph\subparagraph
  \renewcommand{\subparagraph}{
    \@ifstar
      \xxxSubParagraphStar
      \xxxSubParagraphNoStar
  }
  \newcommand{\xxxSubParagraphStar}[1]{\oldsubparagraph*{#1}\mbox{}}
  \newcommand{\xxxSubParagraphNoStar}[1]{\oldsubparagraph{#1}\mbox{}}
\fi
\makeatother


\usepackage{longtable,booktabs,array}
\usepackage{calc} % for calculating minipage widths
% Correct order of tables after \paragraph or \subparagraph
\usepackage{etoolbox}
\makeatletter
\patchcmd\longtable{\par}{\if@noskipsec\mbox{}\fi\par}{}{}
\makeatother
% Allow footnotes in longtable head/foot
\IfFileExists{footnotehyper.sty}{\usepackage{footnotehyper}}{\usepackage{footnote}}
\makesavenoteenv{longtable}
\usepackage{graphicx}
\makeatletter
\newsavebox\pandoc@box
\newcommand*\pandocbounded[1]{% scales image to fit in text height/width
  \sbox\pandoc@box{#1}%
  \Gscale@div\@tempa{\textheight}{\dimexpr\ht\pandoc@box+\dp\pandoc@box\relax}%
  \Gscale@div\@tempb{\linewidth}{\wd\pandoc@box}%
  \ifdim\@tempb\p@<\@tempa\p@\let\@tempa\@tempb\fi% select the smaller of both
  \ifdim\@tempa\p@<\p@\scalebox{\@tempa}{\usebox\pandoc@box}%
  \else\usebox{\pandoc@box}%
  \fi%
}
% Set default figure placement to htbp
\def\fps@figure{htbp}
\makeatother


% definitions for citeproc citations
\NewDocumentCommand\citeproctext{}{}
\NewDocumentCommand\citeproc{mm}{%
  \begingroup\def\citeproctext{#2}\cite{#1}\endgroup}
\makeatletter
 % allow citations to break across lines
 \let\@cite@ofmt\@firstofone
 % avoid brackets around text for \cite:
 \def\@biblabel#1{}
 \def\@cite#1#2{{#1\if@tempswa , #2\fi}}
\makeatother
\newlength{\cslhangindent}
\setlength{\cslhangindent}{1.5em}
\newlength{\csllabelwidth}
\setlength{\csllabelwidth}{3em}
\newenvironment{CSLReferences}[2] % #1 hanging-indent, #2 entry-spacing
 {\begin{list}{}{%
  \setlength{\itemindent}{0pt}
  \setlength{\leftmargin}{0pt}
  \setlength{\parsep}{0pt}
  % turn on hanging indent if param 1 is 1
  \ifodd #1
   \setlength{\leftmargin}{\cslhangindent}
   \setlength{\itemindent}{-1\cslhangindent}
  \fi
  % set entry spacing
  \setlength{\itemsep}{#2\baselineskip}}}
 {\end{list}}
\usepackage{calc}
\newcommand{\CSLBlock}[1]{\hfill\break\parbox[t]{\linewidth}{\strut\ignorespaces#1\strut}}
\newcommand{\CSLLeftMargin}[1]{\parbox[t]{\csllabelwidth}{\strut#1\strut}}
\newcommand{\CSLRightInline}[1]{\parbox[t]{\linewidth - \csllabelwidth}{\strut#1\strut}}
\newcommand{\CSLIndent}[1]{\hspace{\cslhangindent}#1}



\setlength{\emergencystretch}{3em} % prevent overfull lines

\providecommand{\tightlist}{%
  \setlength{\itemsep}{0pt}\setlength{\parskip}{0pt}}



 


\KOMAoption{captions}{tableheading}
\makeatletter
\@ifpackageloaded{caption}{}{\usepackage{caption}}
\AtBeginDocument{%
\ifdefined\contentsname
  \renewcommand*\contentsname{Table of contents}
\else
  \newcommand\contentsname{Table of contents}
\fi
\ifdefined\listfigurename
  \renewcommand*\listfigurename{List of Figures}
\else
  \newcommand\listfigurename{List of Figures}
\fi
\ifdefined\listtablename
  \renewcommand*\listtablename{List of Tables}
\else
  \newcommand\listtablename{List of Tables}
\fi
\ifdefined\figurename
  \renewcommand*\figurename{Figure}
\else
  \newcommand\figurename{Figure}
\fi
\ifdefined\tablename
  \renewcommand*\tablename{Table}
\else
  \newcommand\tablename{Table}
\fi
}
\@ifpackageloaded{float}{}{\usepackage{float}}
\floatstyle{ruled}
\@ifundefined{c@chapter}{\newfloat{codelisting}{h}{lop}}{\newfloat{codelisting}{h}{lop}[chapter]}
\floatname{codelisting}{Listing}
\newcommand*\listoflistings{\listof{codelisting}{List of Listings}}
\makeatother
\makeatletter
\makeatother
\makeatletter
\@ifpackageloaded{caption}{}{\usepackage{caption}}
\@ifpackageloaded{subcaption}{}{\usepackage{subcaption}}
\makeatother
\usepackage{bookmark}
\IfFileExists{xurl.sty}{\usepackage{xurl}}{} % add URL line breaks if available
\urlstyle{same}
\hypersetup{
  pdftitle={Abalone Age Prediction based on Physical Measurements and Sex},
  pdfauthor={Team31: Yuting Ji, Gurveer Madurai, Seungmyun Park \& William Chong},
  colorlinks=true,
  linkcolor={blue},
  filecolor={Maroon},
  citecolor={Blue},
  urlcolor={Blue},
  pdfcreator={LaTeX via pandoc}}


\title{Abalone Age Prediction based on Physical Measurements and Sex}
\author{Team31: Yuting Ji, Gurveer Madurai, Seungmyun Park \& William
Chong}
\date{2025-12-06}
\begin{document}
\maketitle

\renewcommand*\contentsname{Table of contents}
{
\hypersetup{linkcolor=}
\setcounter{tocdepth}{2}
\tableofcontents
}

\subsection{1. Summary}\label{summary}

In this project, we aimed to build a regression model using the
k-Nearest Neighbours (k-NN) algorithm to predict the age of an abalone
using its physical characteristics and sex. Because determining abalone
age traditionally requires cutting the shell and counting internal
rings---a destructive and time-consuming process---machine learning
methods offer a valuable non-destructive alternative for estimating age
from easily measurable features. In the dataset used here, age is
represented as Rings + 1.5, where the additional 1.5 years accounts for
early developmental stages. For example, an abalone with 10 rings would
be approximately 11.5 years old.

Our fitted k-NN regressor (with k = 5) achieved a training RMSE of 1.86
and a test RMSE of 2.29 rings. This means that on unseen data, the
model's predictions typically deviate from the true age by about 2.29
rings. The Figure~\ref{fig-knn-actual-pred} comparing actual and
predicted ring counts shows a generally increasing trend, but also
substantial spread---especially among older abalones. This suggests that
while k-NN captures the broad relationship between size and age,
predicting exact age remains difficult due to biological variability and
overlapping physical measurements. Still, the results demonstrate that
physical features contain meaningful information about abalone age, and
performance could likely improve with hyperparameter tuning or more
advanced models.

Estimating abalone age is important for marine biology research,
sustainable fisheries management, and conservation efforts, yet the
traditional ring-counting method is destructive and impractical at
scale. A predictive model provides a non-invasive alternative, enabling
age estimation for live animals in the field. However, its accuracy is
limited by natural growth variability and nonlinear developmental
patterns, particularly in older individuals. As a result, while machine
learning can support broader age-structure assessments and population
monitoring, it may not fully replace precise biological aging methods in
contexts where exact age determination is required.

\subsection{2. Introduction}\label{introduction}

\paragraph{2.1 Project Goal:}\label{project-goal}

The goal of this project is to investigate whether physical features and
sex can accurately predict the age of an abalone.

\paragraph{2.2 Background:}\label{background}

Abalones are marine molluscs that are widely harvested for food and
shell products. Understanding their age structure is crucial for
fisheries management and conservation planning (Nash et al. 1994).
However, determining age is not straightforward: the most accurate
method requires cutting the shell and counting internal rings under a
microscope, which is destructive, labour-intensive, and not feasible at
scale for monitoring wild populations.

The Abalone dataset from the UCI Machine Learning Repository (Dua and
Graff 1995) provides measurements of physical characteristics including:

\begin{itemize}
\item
  Categorical: sex (M, F, I)
\item
  Continuous: length, diameter, height, whole weight, shucked weight,
  viscera weight, shell weight
\end{itemize}

The dataset contains 4177 observations and 8 predictor variables,
providing a reasonably large sample for training and evaluating
regression models. Age is recorded via the Rings variable, with age in
years calculated as Rings + 1.5. Because Rings is numeric, predicting
age is naturally formulated as a regression problem. Using these
non-destructive measurements to estimate age could support more
sustainable management and research.

\subsection{3. Methods}\label{methods}

\paragraph{3.1 Data and preprocessing}\label{data-and-preprocessing}

Data were downloaded from the UCI repository (Dua and Graff 1995) and
validated using a pandera schema to check column types, allowed
categories for Sex, value ranges, missingness, and duplicate rows. Next,
exploratory data analysis (EDA) is performed using the validated
dataset. This helped to understand the relationships between physical
measurements and age and check for nonlinear patterns and outliers.

Before fitting the model, we performed several preprocessing steps to
ensure data quality and suitability for machine learning. We first
applied one-hot encoding to the Sex variable to convert it into numeric
indicator features. Next, we split the dataset into an 80\% training set
and a 20\% test set to allow for an unbiased evaluation of model
performance. Finally, we used deepchecks to examine potential data
issues, including feature--label correlations, feature--feature
correlations, and label drift between the training and test sets. These
checks helped confirm that our features were appropriate for modeling
and that no major distribution shifts existed between the training and
evaluation data. Highly correlated weight-related features were expected
due to shared biological growth processes, and the checks confirmed that
there were no severe issues beyond this expected multicollinearity.

\paragraph{3.2 Model Selection}\label{model-selection}

We used a k-nearest neighbours (k-NN) regressor to predict abalone age
from physical measurements and sex. For any new abalone, the model
identifies the k most similar individuals in the standardized feature
space, based on Euclidean distance, and predicts age by averaging the
number of rings among those neighbours. We selected k-NN because it is a
non-parametric and flexible method that makes no strong assumptions
about linearity, offers intuitive predictions grounded in similarity to
known examples, and is well suited to capturing the nonlinear
relationships between size, weight, sex, and age observed in abalone
growth patterns (Waugh 1995).

\paragraph{3.3 Model Evaluation}\label{model-evaluation}

Model performance is assessed, which loads the saved k-NN model and
scaler, applies them to the training and test sets, and computes the
RMSE for both splits. The script also generates an
actual-versus-predicted plot for the test set
(Figure~\ref{fig-knn-actual-pred}). RMSE is an appropriate evaluation
metric because the target variable, Rings, is continuous, and the metric
penalizes larger errors more severely, making it well suited for
age-estimation tasks where extreme mispredictions are particularly
undesirabl

\subsection{4. Results}\label{results}

\paragraph{4.1 Exploratory Data
Analysis}\label{exploratory-data-analysis}

Figure~\ref{fig-eda-plot} summarizes how each physical measurement
relates to the number of rings, which serves as a proxy for abalone age.
In general, positive associations are observed across most predictors,
indicating that larger and heavier abalones tend to have more rings. The
observed patterns are nonlinear, with growth appearing faster in younger
abalones and gradually slowing for older ones. This widening spread
suggests that abalones with similar sizes can still vary considerably in
age. Weight-based variables show particularly strong correlations with
age, while height appears to be the least informative due to its very
narrow range.

Across all measurements, the three sex categories overlap substantially,
indicating that sex alone does not form clear separations in age
patterns. Because of this, sex is unlikely to serve as a strong
individual predictor at this stage, although it is still included as a
categorical feature for completeness.

\begin{figure}

\centering{

\includegraphics[width=1\linewidth,height=\textheight,keepaspectratio]{../results/eda_scatter_matrix.png}

}

\caption{\label{fig-eda-plot}Comparison of the relationships between
abalone physical measurements and age (Rings)}

\end{figure}%

\paragraph{4.2 Model Performance}\label{model-performance}

To evaluate whether physical characteristics can predict age, we trained
a k-Nearest Neighbours regression model (k = 5) using scaled predictors.
Figure~\ref{fig-knn-actual-pred} shows the relationship between actual
and predicted ring counts on the test set. The dashed line represents
perfect prediction. While the model captures the general increasing
trend, there is considerable spread, especially for older abalones. This
indicates that predictions become less reliable as age increases, which
is consistent with the biological variability seen in the EDA.

\begin{figure}

\centering{

\includegraphics[width=1\linewidth,height=\textheight,keepaspectratio]{../results/knn_eval_plot.png}

}

\caption{\label{fig-knn-actual-pred}Actual vs.~predicted number of rings
for the kNN regression model.}

\end{figure}%

Model accuracy is summarized using RMSE, with a training RMSE of 1.86
and a test RMSE of 2.29. These values are reported in
Table~\ref{tbl-rmse}, which compares model performance on the training
and test sets. Given that abalone age in years is calculated as Rings +
1.5, a test RMSE of 2.29 rings implies that the model typically predicts
age with an error of roughly 2.29 years. This level of accuracy is
reasonable for a simple, non-parametric model and suggests that physical
features provide useful---but imperfect---information about age.

\begin{longtable}[]{@{}lll@{}}

\caption{\label{tbl-rmse}RMSE values for training and test data.}

\tabularnewline

\toprule\noalign{}
& Dataset & RMSE \\
\midrule\noalign{}
\endhead
\bottomrule\noalign{}
\endlastfoot
0 & Train & 1.863 \\
1 & Test & 2.288 \\

\end{longtable}

\subsection{5. Discussion}\label{discussion}

The EDA and modelling results indicate that abalone age is broadly
reflected in its size and weight, although substantial natural
variability and nonlinear growth patterns limit prediction accuracy. The
widening spread in Figure~\ref{fig-knn-actual-pred}, along with the
moderate RMSE scores, suggests that physical measurements become weaker
indicators of age as abalones grow older.

From a fisheries and conservation standpoint, the model is still
valuable because it enables non-destructive age estimation from easily
collected measurements. This makes it useful for population-level
assessments where approximate age ranges are sufficient. However, the
prediction error of roughly two rings may be too large for fine-grained
applications, such as distinguishing between neighbouring age classes
for precise stock assessments.

Several assumptions and limitations must also be acknowledged. The model
assumes that the abalone dataset (Dua and Graff 1995) is representative
of real-world populations, even though abalone growth varies across
different habitats and environments. Many weight-based predictors are
highly correlated, which is biologically expected but can make
distance-based methods like k-NN sensitive to small measurement
fluctuations. Additionally, k-NN cannot extrapolate beyond observed data
and performs best when the training feature space is densely populated
(Clark, Schreter, and Adams 1996).

Future work could include tuning the hyperparameters of the model, such
as the value of k and the choice of distance metric, exploring more
expressive models like random forests or gradient boosting, and
engineering new features that capture biological growth ratios or
nonlinear interactions. Methods that explicitly model the nonlinear
relationship between age and physical size may also improve predictive
performance.

Overall, the current k-NN model provides a reasonable baseline and
demonstrates that physical measurements contain meaningful information
about abalone age. However, additional modelling, feature engineering,
and validation are needed before the model could be applied reliably in
operational or management settings.

\subsection*{6. References:}\label{references}
\addcontentsline{toc}{subsection}{6. References:}

\phantomsection\label{refs}
\begin{CSLReferences}{1}{0}
\bibitem[\citeproctext]{ref-clark1996}
Clark, D., Z. Schreter, and A. Adams. 1996. {``A Quantitative Comparison
of Dystal and Backpropagation.''} In \emph{Proceedings of the Australian
Conference on Neural Networks (ACNN'96)}.

\bibitem[\citeproctext]{ref-dua1995}
Dua, Dheeru, and Casey Graff. 1995. {``Abalone Data Set.''} UCI Machine
Learning Repository.
\url{https://archive.ics.uci.edu/dataset/1/abalone}.

\bibitem[\citeproctext]{ref-nash1994}
Nash, W. J., T. L. Sellers, S. R. Talbot, A. J. Cawthorn, and W. B.
Ford. 1994. {``The Population Biology of Abalone (Haliotis Species) in
Tasmania. I. Blacklip Abalone (h. Rubra) from the North Coast and
Islands of Bass Strait.''} Technical Report No. 48. Sea Fisheries
Division.

\bibitem[\citeproctext]{ref-waugh1995}
Waugh, S. 1995. {``Extending and Benchmarking Cascade-Correlation.''}
PhD thesis, Department of Computer Science, University of Tasmania.

\end{CSLReferences}




\end{document}
